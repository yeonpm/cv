% !TeX program = xelatex
\documentclass[a4paper, 11pt]{article}

% --------------------------------------------------
% 패키지 설정
% --------------------------------------------------
\usepackage{kotex}              % 한글 지원
\usepackage[left=2cm,right=2cm,top=2cm,bottom=2cm]{geometry} % 여백 설정
\usepackage{tabularx}           % 레이아웃 (표)
\usepackage{graphicx}           % 이미지
\usepackage{hyperref}           % 하이퍼링크
\usepackage{enumitem}           % 리스트 스타일
\usepackage{xcolor}             % 색상
\usepackage{titlesec}           % 섹션 스타일
\usepackage{fontspec}           % 폰트 설정

% --------------------------------------------------
% 폰트 설정 (고딕체)
% --------------------------------------------------
% Windows 사용자의 경우 '맑은 고딕'이 가장 무난합니다.
\setmainfont{Malgun Gothic}
\setmainhangulfont{Malgun Gothic}
\setsansfont{Malgun Gothic}

% --------------------------------------------------
% 커스텀 명령어 정의
% --------------------------------------------------
\hypersetup{
    colorlinks=true,
    linkcolor=black,
    filecolor=black,
    urlcolor=blue,
}

\newcommand{\csection}[2]{
    \vspace{12pt}
    {\Large\textbf{#1}} \hrulefill \\ \vspace{3pt}
    #2
}

\newcommand{\clink}[1]{
    \vspace{2pt}
    {\small #1}
}

% 하위 섹션 스타일 (Work Experience, Personal Projects 구분용)
\newcommand{\csubsection}[1]{
    \vspace{5pt}
    {\large \textbf{#1}} \\ \vspace{2pt}
}

% --------------------------------------------------
% 문서 시작
% --------------------------------------------------
\begin{document}

\noindent
% --------------------------------------------------
% 헤더 (프로필 정보)
% --------------------------------------------------
\begin{tabularx}{\linewidth}{@{} X r @{}}
    {
        {\Huge \textbf{김 재 연}} \newline \vspace{5pt}
        
        {\large 프론트엔드 개발자 (Frontend Developer)} \newline \vspace{5pt}

        \clink{
            \href{https://github.com/yeonpm}{github.com/yeonpm} \textbf{·}
            \href{mailto:yeonpmdevelop@gmail.com}{yeonpmdevelop@gmail.com} \newline
        }
    }
    & 
    % 프로필 이미지
    {
        \includegraphics[width=3.5cm, height=4.5cm]{profile_image.jpg}
    }
\end{tabularx}

\vspace{10pt}

% --------------------------------------------------
% Summary
% --------------------------------------------------
\csection{SUMMARY}{
    \begin{itemize}[leftmargin=*, noitemsep]
        \item 2020년부터 React, React Native, Next.js, TypeScript를 주력으로 사용하는 프론트엔드 개발자입니다.
        \item 2019년도에 앱서비스 창업을 시작으로 개발을 시작하였고, 하나의 제품을 A to Z 개발하며 얻게 된 비즈니스 로직의 관점과 사용자 경험의 중요성, 문제 해결을 우선시 하며 개발합니다.
        \item 여러 프로젝트들을 할 때마다 반복적인 작업을 개선하기위해 몇 가지 NPM 라이브러리화하여 배포하고, 주식회사 핸들에서 다양한 웹/앱 서비스를 런칭한 경험이 있습니다.
        \item Python, REST API 활용 능력과 백엔드(Firebase, Node.js)에 대한 이해를 바탕으로 폭 넓은 개발이 가능하며, 문제 해결을 위해서 다각도로 고민합니다.
        \item 새로운 기술을 항상 팔로우업 하며, 효율적인 코드 작성과 사용자 경험 개선에 집중합니다.
    \end{itemize}
}

% --------------------------------------------------
% Skills
% --------------------------------------------------
\csection{SKILLS}{
    \begin{itemize}[leftmargin=*, noitemsep]
        \item \textbf{Frontend}: React, React Native, Next.js, TypeScript, Zustand, MobX, Redux
        \item \textbf{Backend \& Cloud}: Python, Node.js, Java, Firebase, AWS (EC2, Route53), Docker
        \item \textbf{Tools}: GitHub, Bitbucket, Jira, Figma, Zeplin, Slack, Notion, Cursor
        \item \textbf{Others}: GitHub Actions, Bitbucket Pipeline, Llama Large Language Model, YOLO Computer Vision, Google Adsense, Google Analytics, Vercel
    \end{itemize}
}

% --------------------------------------------------
% Projects & Experience
% --------------------------------------------------
\csection{PROJECTS \& EXPERIENCE}{
    
    % 1. Work Experience (회사 업무)
    \csubsection{Work Experience}
    \begin{itemize}[leftmargin=*, itemsep=5pt]
        \item \textbf{주식회사 핸들 (Handle)} \hfill 2020 -- 현재 \\
        \textit{Frontend Lead / Full-Stack Developer}
        \begin{itemize}[label={-}, noitemsep, topsep=2pt]
            
            \item \textbf{GUI 기반 차량 데이터 맵핑 어드민 개발} (React)
                \begin{itemize}[label=$\cdot$, noitemsep]
                    \item 일 200~300대 등록되는 차량의 트림 맵핑 작업 시간을 대당 2분 $\rightarrow$ \textbf{2초}로 단축 (약 98\% 효율 개선)
                    \item 반복 업무 자동화 로직과 직관적인 UI 제공으로 운영팀 업무 효율성 극대화
                \end{itemize}

            \item \textbf{파트너스 웹(딜러용 매매 관리 시스템) 개발}
                \begin{itemize}[label=$\cdot$, noitemsep]
                    \item 회사의 첫 메인 프로젝트로, 프로젝트 초기 아키텍처 설계부터 테이블 등 핵심 UI 컴포넌트 라이브러리 직접 구현 및 운영
                \end{itemize}

            \item \textbf{차량 이미지 합성 및 표준화 모듈 개발} (Python, Computer Vision)
                \begin{itemize}[label=$\cdot$, noitemsep]
                    \item 고비용(약 1.7억)의 AI 아웃소싱 결과물 한계를 극복하기 위해 단독으로 이미지 처리 모듈 설계 및 개발
                    \item 배경 제거(Segment), 번호판 인식/마스킹, 그림자 합성 로직 구현으로 플랫폼 이미지 통일성 확보 및 브랜드 신뢰도 제고
                \end{itemize}

            \item \textbf{중고차 시세 산출 알고리즘 개발} (Python, RANSAC)
                \begin{itemize}[label=$\cdot$, noitemsep]
                    \item 사내 실거래 데이터를 기반으로 RANSAC 알고리즘을 적용, 이상치(Outlier)를 제거한 고정밀 시세 로직 구현
                    \item 경쟁사 대비 높은 정확도의 시세 정보를 제공하여 매입/판매 의사결정 지원 (백엔드 개발자와 협업)
                \end{itemize}
            
            \item \textbf{오토허브셀카 ERP 시스템 구축} (React, TypeScript)
                \begin{itemize}[label=$\cdot$, noitemsep]
                    \item 100개 이상의 메뉴와 복잡한 기능을 포함한 대규모 ERP 프론트엔드 단독 개발
                    \item 백엔드 개발자 8명과의 협업 환경에서 확장성 있는 컴포넌트 구조 설계로 안정적인 서비스 런칭
                \end{itemize}
        \end{itemize}
    \end{itemize}

    \vspace{5pt}

    % 2. Personal Projects (개인 프로젝트)
    \csubsection{Personal Projects}
    \begin{itemize}[leftmargin=*, itemsep=5pt]
        \item \textbf{LLM Fine-tuning & Research} \hfill 2023 \\
        \textit{한국방사성폐기물학회 우수포스터상 수상}
        \begin{itemize}[label={-}, noitemsep]
            \item Llama 4 모델을 도메인 특화 데이터로 Fine-tuning하여 전문 지식 질의응답 모델 개발
            \item NVIDIA H100 GPU 환경에서 Python(PyTorch, Hugging Face)을 활용하여 모델 학습 및 최적화 수행
            \item 프론트엔드 개발자임에도 최신 AI 기술(LLM)을 빠르게 습득하고 실제 연구 성과로 연결하는 실행력 입증
        \end{itemize}

        \item \textbf{NPM Libraries Development} \hfill Open Source \\
        \textit{React 및 JS 모듈 라이브러리 개발 및 배포}
        \begin{itemize}[label={-}, noitemsep]
            \item \textbf{@yeonpm/react}: React 커스텀 훅 및 UI 유틸리티 라이브러리 (\href{https://www.npmjs.com/package/@yeonpm/react}{NPM Link})
            \item \textbf{react-style-props}: React 컴포넌트 스타일링을 위한 유틸리티 라이브러리 (\href{https://www.npmjs.com/package/react-style-props}{NPM Link})
            \item \textbf{yeonpm-modules}: 자주 사용하는 JavaScript 모듈 모음 (\href{https://www.npmjs.com/package/yeonpm-modules}{NPM Link})
        \end{itemize}

        \item \textbf{Web Services \& Applications} \hfill Personal Projects \\
        \textit{풀스택 웹 서비스 설계 및 개발}
        \begin{itemize}[label={-}, noitemsep, topsep=2pt]
            \item \textbf{Rgbtion} (\href{https://rgbtion.vercel.app}{Link}): 미술 작품 이커머스 플랫폼 개발 (Next.js, Vercel)
            \item \textbf{Prinprin} (\href{https://prinprin.vercel.app}{Link}): 커스텀 프린팅 티셔츠 제작 서비스
            \item \textbf{Kholidayz} (\href{https://www.kholidayz.com}{Link}): 한국 공휴일 정보를 제공하는 달력 서비스
            \item \textbf{Ambiguous Machines} (\href{https://www.ambiguousmachines.com}{Link}): 인터랙티브 웹 플래시 게임 구현
        \end{itemize}

        \item \textbf{Automated Trading \& Tools} \hfill \\
        \textit{자동화 프로그램 및 유틸리티 개발}
        \begin{itemize}[label={-}, noitemsep, topsep=2pt]
            \item \textbf{Renko strikes}: 주식 자동 매매 프로그램 개발 (Python, KIS API 활용)
            \item \textbf{Tenthirty}: 할인 결제 앱 서비스 개발
        \end{itemize}
    \end{itemize}
}

\end{document}
