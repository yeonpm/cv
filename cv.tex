% !TeX program = xelatex
\documentclass[a4paper, 11pt]{article}

% --------------------------------------------------
% 패키지 설정
% --------------------------------------------------
\usepackage{kotex}              % 한글 지원
\usepackage[left=2cm,right=2cm,top=2cm,bottom=2cm]{geometry} % 여백 설정
\usepackage{tabularx}           % 레이아웃 (표)
\usepackage{graphicx}           % 이미지
\usepackage{hyperref}           % 하이퍼링크
\usepackage{enumitem}           % 리스트 스타일
\usepackage{xcolor}             % 색상
\usepackage{titlesec}           % 섹션 스타일
\usepackage{fontspec}           % 폰트 설정

% --------------------------------------------------
% 폰트 설정 (고딕체)
% --------------------------------------------------
% Windows 사용자의 경우 '맑은 고딕'이 가장 무난합니다.
% macOS 사용자의 경우 'Apple SD Gothic Neo'를 사용합니다.
\setmainfont{Apple SD Gothic Neo}
\setmainhangulfont{Apple SD Gothic Neo}
\setsansfont{Apple SD Gothic Neo}

% --------------------------------------------------
% 커스텀 명령어 정의
% --------------------------------------------------
\hypersetup{
    colorlinks=true,
    linkcolor=black,
    filecolor=black,
    urlcolor=blue,
}

\newcommand{\csection}[2]{
    \vspace{12pt}
    {\Large\textbf{#1}} \hrulefill \\ \vspace{3pt}
    #2
}

\newcommand{\clink}[1]{
    \vspace{2pt}
    {\small #1}
}

% 하위 섹션 스타일 (Work Experience, Personal Projects 구분용)
\newcommand{\csubsection}[1]{
    \vspace{5pt}
    {\large \textbf{#1}} \\ \vspace{2pt}
}

% --------------------------------------------------
% 문서 시작
% --------------------------------------------------
\begin{document}

% --------------------------------------------------
% 헤더 (프로필 정보)
% --------------------------------------------------
\vspace*{-0.5cm}
\noindent
\begin{tabularx}{\linewidth}[t]{@{} X r @{}}
    \vspace{-\topskip}\vspace{-\baselineskip}\vspace{20pt}
    {\Huge \textbf{김 재 연}} \newline \vspace{5pt}
    
    {\large 프론트엔드 개발자 (Frontend Developer)} \newline \vspace{5pt}

    \clink{
        \href{https://github.com/yeonpm}{github.com/yeonpm} \textbf{·}
        \href{mailto:yeonpmdevelop@gmail.com}{yeonpmdevelop@gmail.com} \newline
    }
    & 
    % 프로필 이미지
    \vspace{-\topskip}\vspace{-\baselineskip}\vspace{-0.5cm}
    \raisebox{-100pt}{\includegraphics[width=3.5cm, height=4.5cm]{profile_image_250.jpg}}
\end{tabularx}

\vspace{60pt}

% --------------------------------------------------
% Summary
% --------------------------------------------------
\csection{SUMMARY}{
    \begin{itemize}[leftmargin=*, noitemsep]
        \item[] \textbf{비즈니스 가치를 창출하고 기술로 문제를 해결하는 프론트엔드 엔지니어} \\
        \item \textbf{Full-cycle Development}: 창업(A to Z)부터 대규모 ERP 단독 개발까지, \textbf{서비스 기획부터 배포, 운영까지 주도적으로 리딩} 가능한 경험을 보유하고 있습니다.
        \item \textbf{Performance Optimization}: 단순 기능 구현을 넘어, 반복 업무를 98\% 단축하거나 고비용의 외부 솔루션을 자체 기술로 대체(1.7억 절감)하는 등 \textbf{비즈니스 임팩트}를 최우선으로 고려합니다.
        \item \textbf{Technical Depth \& Expansion}: React 생태계(React Native, Next.js)를 주력으로 하며, Python/CV/LLM 등 필요한 기술이라면 영역을 가리지 않고 습득하여 \textbf{문제 해결의 도구}로 활용합니다.
        \item \textbf{Engineering Culture}: 반복되는 작업을 라이브러리화(NPM)하여 생산성을 높이고, 습득한 지식을 팀 내에 적극적으로 전파하여 \textbf{함께 성장하는 문화}를 지향합니다.
    \end{itemize}
}

% --------------------------------------------------
% Skills
% --------------------------------------------------
\csection{SKILLS}{
    \begin{itemize}[leftmargin=*, noitemsep]
        \item \textbf{Frontend}: React, React Native, Next.js, TypeScript, Zustand, MobX, Redux
        \item \textbf{Backend \& Cloud}: Python, Node.js, Java, Firebase, Supabase, MySQL, AWS (EC2, Route53), Docker
        \item \textbf{Tools}: GitHub, Bitbucket, Jira, Figma, Zeplin, Slack, Notion, Cursor
        \item \textbf{Others}: GitHub Actions, Bitbucket Pipeline, Llama Large Language Model, YOLO Computer Vision, Google Adsense, Google Analytics, Vercel
    \end{itemize}
}

% --------------------------------------------------
% Projects & Experience
% --------------------------------------------------
\csection{PROJECTS \& EXPERIENCE}{
    
    % 1. Work Experience (회사 업무)
    \csubsection{Work Experience}
    \begin{itemize}[leftmargin=*, itemsep=5pt]
        \item \textbf{주식회사 핸들 (Handle)} \hfill 2021 -- 현재 \\
        \begin{itemize}[label={-}, noitemsep, topsep=2pt]
            
            \item \textbf{GUI 기반 차량 데이터 맵핑 어드민 개발} (React) \hfill 2021 2Q
                \begin{itemize}[label=$\cdot$, noitemsep]
                    \item 운영팀의 반복 업무 비효율을 개선하기 위해 주도적으로 GUI 기반 맵핑 시스템 기획 및 개발
                    \item GraphQL을 도입하여 프론트엔드 단에서의 검색 로직을 강화하고, 직관적인 UI/UX로 업무 프로세스 최적화
                    \item 일 200\textasciitilde 300대 등록되는 차량의 트림 맵핑 작업 시간을 대당 2분 $\rightarrow$ \textbf{2초}로 단축 (약 98\% 효율 개선)
                \end{itemize}

            \item \textbf{파트너스 웹(딜러용 매매 관리 시스템) 개발} (React, Next.js, TypeScript) \hfill 2021 3Q - 2022 1Q
                \begin{itemize}[label=$\cdot$, noitemsep]
                    \item 회사의 첫 메인 프로젝트로, 프로젝트 초기 아키텍처 설계부터 테이블 등 핵심 UI 컴포넌트 라이브러리 직접 구현 및 운영, 다양한 화면 개발
                \end{itemize}

            \item \textbf{유저 앱/웹 개발} (React Native, React, Next.js, TypeScript) \hfill 2022 2Q - 2023 1Q
                \begin{itemize}[label=$\cdot$, noitemsep]
                    \item 회사의 핵심 커머스 플랫폼으로 4인 프론트엔드 팀으로 개발
                    \item 모바일앱/웹 사용자 모두를 고려한 웹뷰 기반 앱 개발
                    \item react-native 및 react 주요 컴포넌트 개발
                    \item 검색 필터, 검색 리스트 화면, 대출 및 현금 결제 화면, 문의 화면 등 다양한 핵심 화면 개발
                    \item 운영 배포를 위한 빌드 자동화 스크립트 개발
                    \item 차량 이미지 렌더링속도 개선 및 최적화
                \end{itemize}

            \item \textbf{차량 이미지 합성 및 표준화 모듈 개발} (Python, Yolo Computer Vision) \hfill 2023 1Q - 2023 4Q
                \begin{itemize}[label=$\cdot$, noitemsep]
                    \item 프론트엔드 사용자 경험(UX) 향상과 이미지 로딩 속도 최적화를 위해 백엔드 이미지 처리 파이프라인을 주도적으로 설계 및 개발 (외주개발비용 약 1.7억 절감)
                    \item 배경 제거, 번호판 마스킹, 그림자 합성 로직 구현으로 플랫폼 이미지 통일성 확보 및 브랜드 신뢰도 제고
                    \item 기존 평균 2MB의 PNG 이미지를 동일 퀄리티로 100KB 내외의 WEBP 포맷으로 경량화하여 웹 성능(LCP) 개선 및 스토리지 비용 절감
                    \item 2-3\%의 이미지 합성 오류를 보완할 수 있는 수동 합성 프로그램 개발 (Next.js, TypeScript, Slack API)
                \end{itemize}

            \item \textbf{중고차 시세 산출 알고리즘 개발} (Python, RANSAC) \hfill 2023 4Q
                \begin{itemize}[label=$\cdot$, noitemsep]
                    \item 사내 실거래 데이터를 기반으로 RANSAC 알고리즘을 적용, 이상치(Outlier)를 제거한 고정밀 시세 로직 구현
                    \item 핵심 로직을 단독으로 맡아서 개발하여 자체 개발한 알고리즘으로 중고차 가격을 high/normal/great/excellent 네 단계로 평가하여 사용자에게 제공
                    \item 경쟁사 대비 높은 정확도의 시세 정보를 제공하여 사용자에게 매입/판매 의사결정 지원
                \end{itemize}
            
            \item \textbf{오토허브셀카 ERP 시스템 개발} (React, TypeScript) \hfill 2024 1Q - 2025 4Q
                \begin{itemize}[label=$\cdot$, noitemsep]
                    \item 110개 이상의 메뉴와 복잡한 기능을 포함한 대규모 ERP 프론트엔드 단독 개발
                    \item 유료 테이블 라이브러리를 대체하기 위해 가상화(Virtualization) 기술을 적용한 고성능 데이터 그리드 자체 개발
                    \item 복잡한 필터링 및 커스텀 기능 요구사항을 충족하는 유연한 컴포넌트 설계로 렌더링 성능 최적화 및 확장성 확보
                    \item 백엔드 개발자 7명과의 협업 환경에서 확장성 있는 컴포넌트 구조 설계로 안정적인 서비스 런칭
                \end{itemize}
        \end{itemize}
    \end{itemize}

    \vspace{5pt}

    % 2. Personal Projects (개인 프로젝트)
    \csubsection{Personal Projects}
    \begin{itemize}[leftmargin=*, itemsep=5pt]
        \item \textbf{모바일 앱 기반 서비스 창업 프로젝트 TENTHIRTY} \hfill 2019 3Q - 2021 1Q \\
        \textit{TENTHIRTY라는 사업자로 할인 결제 앱 서비스 풀스택 개발}
        \begin{itemize}[label={-}, noitemsep, topsep=2pt]
            \item React Native, JSP, java, MySQL 기반으로 특허출원, 영업, 유저앱 및 파트너앱 출시 등 2인 프로젝트로 A to Z 진행
            \item 이 당시 독학으로 개발을 배우고 적용하게 되면서 폭 넓은 앱/웹 개발 경험을 축적했습니다.
        \end{itemize}

        \item \textbf{LLM Fine-tuning Research} \hfill 2025 1Q - 2025 2Q \\
        \textit{한국방사성폐기물학회(KRS) 우수포스터상 수상 (Automated MCNP Input Deck Generation using Llama
        : A Deep Learning Approach) \href{https://www.krs.or.kr/html/?pmode=academic_societies }{Link}}
        \begin{itemize}[label={-}, noitemsep]
            \item Meta의 opensource Large Language Model인 Llama 4 모델을 원자력공학 도메인 특화 데이터로 Fine-tuning하여 설계 코드 생성 모델 개발
            \item NVIDIA H100 x8 GPU 환경에서 Python(PyTorch, Hugging Face)을 활용하여 모델 학습 및 최적화 수행
            \item 프론트엔드 개발자임에도 최신 AI 기술(LLM)을 빠르게 습득하고 실제 연구 성과로 연결하는 실행력 입증
        \end{itemize}

        \item \textbf{NPM Libraries Development} \hfill Open Source \\
        \textit{React 및 JS 모듈 라이브러리 개발 및 배포}
        \begin{itemize}[label={-}, noitemsep]
            \item \textbf{@yeonpm/react}: React 커스텀 훅 및 UI 유틸리티 라이브러리 (\href{https://www.npmjs.com/package/@yeonpm/react}{NPM Link})
            \item \textbf{react-style-props}: React 컴포넌트 스타일링을 위한 유틸리티 라이브러리 (\href{https://www.npmjs.com/package/react-style-props}{NPM Link})
            \item \textbf{yeonpm-modules}: 자주 사용하는 JavaScript 모듈 모음 (\href{https://www.npmjs.com/package/yeonpm-modules}{NPM Link})
        \end{itemize}

        \item \textbf{Web Services \& Applications} \hfill Personal Projects \\
        \textit{풀스택 웹 서비스 설계 및 개발 (사이드 프로젝트)}
        \begin{itemize}[label={-}, noitemsep, topsep=2pt]
            \item \textbf{Kholidayz} (\href{https://www.kholidayz.com}{Link}): 한국 공휴일 정보를 제공하는 달력 사이트 (사이드 프로젝트) (Next.js, Vercel)
            \item \textbf{Rgbtion} (\href{https://rgbtion.vercel.app}{Link}): 미술 작품 이커머스 플랫폼 (사이드 프로젝트) (Next.js, Vercel, firebase database storage auth)
            \item \textbf{Prinprin} (\href{https://prinprin.vercel.app}{Link}): 커스텀 프린팅 티셔츠 제작 서비스 (사이드 프로젝트) (Next.js, Vercel)
            \item \textbf{Ambiguous Machines} (\href{https://ambiguousmachines.vercel.app}{Link}): 인터랙티브 웹 플래시 게임 구현 (사이드 프로젝트) (Next.js, Vercel, Three.js, firebase database)
        \end{itemize}
    \end{itemize}
}

\end{document}
